\documentclass[12pt,a4paper]{article}
\usepackage{etex,datetime,setspace,latexsym,amssymb,amsmath,amsthm}
\usepackage{fancybox,dialogue,float,wrapfig,enumerate,microtype}
\usepackage{verbatim,xcolor,multicol,titlesec,tabularx,mdframed}

\usepackage[utf8]{inputenc}
\usepackage[style=authoryear]{biblatex}
\addbibresource{references.bib}
% \bibliographystyle{alpha}
\usepackage[pdftex]{hyperref}
\usepackage[margin=2cm,bottom=3cm,footskip=15mm]{geometry}
\parindent0cm
\parskip0.5em

\usepackage{enumitem}
\setlist[itemize]{itemsep=1pt, topsep=2pt} % Adjust 5pt to your preferred spacing


\usepackage{tikz}
\usetikzlibrary{arrows,trees,positioning,shapes,patterns}
\usetikzlibrary{intersections,calc,fpu,decorations.pathreplacing}

\usepackage[T1]{fontenc} % better fonts

% Haskell code listings in our own style
\usepackage{listings,color}
\definecolor{lightgrey}{gray}{0.35}
\definecolor{darkgrey}{gray}{0.20}
\definecolor{lightestyellow}{rgb}{1,1,0.92}
\definecolor{dkgreen}{rgb}{0,.2,0}
\definecolor{dkblue}{rgb}{0,0,.2}
\definecolor{dkyellow}{cmyk}{0,0,.7,.5}
\definecolor{lightgrey}{gray}{0.4}
\definecolor{gray}{gray}{0.50}
\lstset{
  language        = Haskell,
  basicstyle      = \scriptsize\ttfamily,
  keywordstyle    = \color{dkblue},     stringstyle     = \color{red},
  identifierstyle = \color{dkgreen},    commentstyle    = \color{gray},
  showspaces      = false,              showstringspaces= false,
  rulecolor       = \color{gray},       showtabs        = false,
  tabsize         = 8,                  breaklines      = true,
  xleftmargin     = 8pt,                xrightmargin    = 8pt,
  frame           = single,             stepnumber      = 1,
  aboveskip       = 2pt plus 1pt,
  belowskip       = 8pt plus 3pt
}
\lstnewenvironment{code}[0]{}{}

% only shown, not compiled:
\lstnewenvironment{showCode}[0]{\lstset{numbers=none}}{}

% only compiled, not shown:
\newcommand{\hide}[1]{}

% will the real phi please stand up
\renewcommand{\phi}{\varphi}

% load hyperref as late as possible for compatibility
\usepackage[pdftex]{hyperref}
\hypersetup{
  pdfborder = {0 0 0},
  breaklinks = true,
  linktoc = all,
}
\pdfinfoomitdate=1
\pdftrailerid{}
\pdfsuppressptexinfo15



\title{$\mathcal{KL}$ as a Knowledge Base Logic in Haskell}
\author{Natasha De Kriek, Milan Hartwig, Victor Joss, Paul Weston, Louise Wilk}
\date{\today}
\hypersetup{pdfauthor={Me}, pdftitle={My Report}}
\addbibresource{references.bib}
\begin{document}

\maketitle

\begin{abstract}
In this project, we aim to implement the first-order epistemic logic $\mathcal{KL}$ as introduced by \textcite{levesque1981} and refined by \textcite{Lokb}. 
The semantics for this logic evaluates formulae on world states and epistemic states where world states are sets of formulae that are true at the world and epistemic states are sets of world states that are epistemically accessible. Levesque and Lakemeyer use the language $\mathcal{KL}$ as ``a way of communicating with a knowledge base'' (ibid. p. 79). For this, they define an ASK- and a TELL-operation on a knowledge base. In our project, we implement a  $\mathcal{KL}$-model, the ASK- and TELL- operations, a tableau-based satisfiablity and validity checking for  $\mathcal{KL}$, as well as compare  $\mathcal{KL}$-models to epistemic Kripke models and implement a translation function between them.
\end{abstract}

% \vfill

\tableofcontents

\clearpage

% We include one file for each section. The ones containing code should
% be called something.lhs and also mentioned in the .cabal file.


\section{\texorpdfstring{$\mathcal{KL}$}{KL}: Syntax and Semantics}\label{sec:KLmodel}

\input{lib/SyntaxKL.lhs}

\input{lib/SemanticsKL.lhs}


\input{lib/AskTell.lhs}

% MISSING TRANSLATION MODULE
\input{lib/Translator.lhs}

\input{lib/Tableau.lhs}

\subsection{Tests}
You can run all the tests and examine the current code coverage run \verb|stack clean && stack test --coverage|.

% importing tests
\input{test/Generators.lhs}
\input{test/AskTellSpec.lhs}
\input{test/SemanticsKLSpec.lhs}
\input{test/SyntaxKLSpec.lhs}
\input{test/TableauSpec.lhs}
\input{test/TranslatorSpec.lhs}

%
\section{Conclusion}\label{sec:Conclusion}

Finally, we can see that \cite{liuWang2013:agentTypesHLPE} is a nice paper.


\addcontentsline{toc}{section}{Bibliography}
% \bibliographystyle{alpha}

\printbibliography

\end{document}
